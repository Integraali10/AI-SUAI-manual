% C-c C-o to insert the block

% Individual equation: equation* block

% Inline equation \begin{math}\frac{sin(x)}{x}\end{math}

\documentclass{article}


\usepackage{amsmath,amssymb}


\ifdefined\ispreview

\usepackage[active,tightpage]{preview}

\PreviewEnvironment{math}

\PreviewEnvironment{equation*}

\fi


\DeclareMathOperator{\E}{\mathbb{E}}

\DeclareMathOperator*{\argmin}{arg\,min}


\begin{document}


Page 1, section ``Value, state and optimality''


We defined value as an expected total reward obtainable from the state. In formal way, value of the state is:

\begin{math}V(s) = \E[\sum_{t=0}^\infty{r_t\gamma^t}]\end{math}, where \begin{math}r_t\end{math} is the local

reward obtained at the step t of the episode.


Page 2, section ``Bellman equation of optimality''


Let’s start with deterministic case, when all our actions have 100\% guaranteed

outcome. Imagine our agent observes state \begin{math}s_0\end{math} and has N

available actions. Every action leads to another state

\begin{math}s_1 \ldots s_N\end{math} with respective reward

\begin{math}r_1\ldots r_N\end{math}. Also assume that we know the

values \begin{math}V_i\end{math} of all states connected to the

state \begin{math}s_0\end{math}.


If we fix the action and calculate the value given to this action, the value

will be \begin{math}V_0(a=a_i) = r_i + V_i\end{math}. So, to choose the best possible action, the agent

needs to calculate resulting values for every action and choose the maximum

possible outcome. In other words: \begin{math}V_0 = \max_{a \in 1 \ldots N}(r_a + V_a)\end{math}. If we’re using discount

factor \begin{math}\gamma\end{math}, we need to multiply value of the next state

by gamma: \begin{math}V_0 = \max_{a \in 1 \ldots N}(r_a + \gamma V_a)\end{math}.




Page 3


It’s not very complicated to extend it for a stochastic case, when our

actions can have chance to end up in different states. What we need to do is to

calculate the expected value for every action instead of just taking the value

of the next state. To illustrate this, let’s consider one single action

available from state \begin{math}s_0\end{math} with three possible outcomes.


Here we have one action available from the state \begin{math}s_0\end{math}, which can lead to three

different states with different probabilities: with probability \begin{math}p_1\end{math} it can end up

in state \begin{math}s_1\end{math}, \begin{math}p_2\end{math} in state \begin{math}s_2\end{math} and

\begin{math}p_3\end{math} in state \begin{math}s_3\end{math} (\begin{math}p_1+p_2+p_3=1\end{math}, of course). Every

target state has its own reward \begin{math}r_1\end{math}, \begin{math}r_2\end{math} or \begin{math}r_3\end{math}. To calculate the expected value

after issuing action 1, we need to sum all values multiplied by their probabilities:



\begin{equation*}

V_0(a=1) = p_1(r_1 + \gamma V_1) + p_2(r_2 + \gamma V_2) + p_3(r_3 + \gamma

V_3)

\end{equation*}


or, more formal


\begin{equation*}

V_0(a) = \E_{s \sim S}[r_{s,a} + \gamma V_s] = \sum_{s\in S} p_{a,0 \rightarrow s}(r_{s,a} + \gamma

V_s)

\end{equation*}


By combining the Bellman

equation for a deterministic case with value for stochastic actions, we get

Bellman equation for general case:


\begin{equation*}

V_0 = \max_{a \in A}\E_{s \sim S}[r_{s,a} + \gamma V_s] = \max_{a \in A} \sum_{s \in S}

p_{a,0 \rightarrow s}(r_{s,a} + \gamma V_s)

\end{equation*}


(Notation \begin{math}p_{a,i \rightarrow j}\end{math} means probability of action a issued in

state i to end up in state j)




Page 4, Value of action


To make our life slightly easier, we can define different quantity in addition

to value of state \begin{math}V_s\end{math}: value of action \begin{math}Q_{s,a}\end{math}. Basically, it equals total

reward we can get by executing action a in state s, and could be defined via

\begin{math}V_s\end{math}. Being much less fundamental entity than \begin{math}V_s\end{math}, this quantity gave a name to

the whole family of methods “Q-learning”, because it is slightly more convenient

in practice. In those methods, our primary objective is to get values of Q for

every pair of state and action.


\begin{equation*}

Q_{s,a} = \E_{s' \sim S}[r_{s,a} + \gamma V_{s'}] =

\sum_{s' \in S} p_{a,s \rightarrow s'}(r_{s,a} + \gamma V_{s'})

\end{equation*}


Which means: Q for this state s and action a equals the expected immediate

reward plus discounted long-term reward of destination state. We also can define

\begin{math}V_s\end{math} via \begin{math}Q_{s, a}\end{math}:


\begin{equation*}

V_s = \max_{a \in A} Q_{s,a}

\end{equation*}


And, finally, we can express Q(s, a) via itself, which will be used in the next chapter’s topic of Q-learning:

\begin{equation*}

Q(s,a)=r_{s,a} + \gamma \max_{a' \in A}Q(s',a')

\end{equation*}


To give you a concrete example, let’s consider a simple environment which is

similar to FrozenLake, but has much simpler structure: we have one initial state

\begin{math}s_0\end{math} surrounded by four target states \begin{math}s_1, s_2, s_3, s_4\end{math} with different rewards.


Let’s calculate the values of actions to begin with. Terminal states \begin{math}s_1 \ldots s_4\end{math} have no

outbound connections, so Q for those states is zero for all actions. Due to this, the

values of the Terminal states are equal to their immediate reward (once we get there,

our episode ends without any subsequent states): \begin{math}V_1 = 1, V_2 = 2, V_3 = 3, V_4 = 4\end{math}.

The values of actions for state 0 are a bit more complicated. Let’s start with the

“up” action. Its value, according to the definition, is equal to the expected sum

of the immediate reward plus long-term value for subsequent steps. We have no

subsequent steps for any possible transition for the “up” action, so

\begin{equation*}

Q(s_0, up) = 0.33 \cdot V_1 + 0.33 \cdot V_2 + 0.33 \cdot V_4 = 0.33 \cdot 1 + 0.33 \cdot 2 + 0.33 \cdot 4 = 2.31

\end{equation*}.


Repeating this for the rest of \begin{math}s_0\end{math} actions results in the following:


\begin{equation*}

Q(s_0, left) = 0.33 \cdot V_1 + 0.33 \cdot V_2 + 0.33 \cdot V_3 = 1.98

\end{equation*}

\begin{equation*}

Q(s_0, right) = 0.33 \cdot V_4 + 0.33 \cdot V_1 + 0.33 \cdot V_3 = 2.64

\end{equation*}

\begin{equation*}

Q(s_0, down) = 0.33 \cdot V_3 + 0.33 \cdot V_2 + 0.33 \cdot V_4 = 2.97

\end{equation*}


The final value for state \begin{math}s_0\end{math} is the maximum of those actions values, which is 2.97.


Page 6, Value iteration method


We start from state \begin{math}s_1\end{math} and the only action we can do leads us to state \begin{math}s_2\end{math}. We

get reward r=1 and the only transition from \begin{math}s_2\end{math} is an action which brings us back

to the \begin{math}s_1\end{math}. So, the life of our agent is an infinite sequence of states [

\begin{math}s_1, s_2, s_1, s_2, s_1, s_2, s_1, s_2,\ldots\end{math}]. To deal with this infinity, we can use a discount

factor \begin{math}\gamma=0.9\end{math}. Now, the question: what’s the values for both states?


The answer is not very complicated, though. Every transition from \begin{math}s_1\end{math} to \begin{math}s_2\end{math} gives

us reward of 1 and every back transition gives us 2. So, our sequence of rewards

will be [1, 2, 1, 2, 1, 2, 1, 2, ….]. As there is only one action available in

every state, our agent has no choice, so, we can omit max operation in formulas

(there is only one alternative). Value for every state will be equal to the

infinite sum:



\begin{equation*}

V(s_1) = 1 + \gamma (2 + \gamma(1 + \gamma(2 + \ldots))) = \sum_{i=0}^\infty 1\gamma^{2i}+2\gamma^{2i+1}

\end{equation*}



\begin{equation*}

V(s_2) = 2 + \gamma (1 + \gamma(2 + \gamma(1 + \ldots))) = \sum_{i=0}^\infty 2\gamma^{2i}+1\gamma^{2i+1}

\end{equation*}


Strictly speaking, we cannot calculate the exact values for our states, but with

\begin{math}\gamma=0.9\end{math}, contribution of every transition quickly decreases over time. For

example, after 10 steps, \begin{math}\gamma^{10} = 0.910 = 0.349\end{math}, but after 100 steps it becomes just

0.0000266. Due to this, we can stop after 50 iterations and still get quite

precise estimation.




\begin{enumerate}

\item Initialize values of all states \begin{math}V_i\end{math} to some

initial value, usually zero.

% todo: check

\item For every state s in the MDP perform Bellman update: \begin{math}V_s \leftarrow

\max_a \sum_{s'}p_{a,s \rightarrow s'}(r_{s,a} + \gamma V_{s'})\end{math}

\item Repeat step 2 for some large amount of steps or until changes become too small.

\end{enumerate}


Only minor modifications to the above procedure are required in case of action values (i.e. Q):



\begin{enumerate}

\item Initialize all \begin{math}Q_{s,a}\end{math} to zero

\item For every state s and every action a in this state perform

update: \begin{math}Q_{s,a} \leftarrow \sum_{s'}p_{a,s \rightarrow s'}(r_{s,a} + \gamma \max_{a'}Q_{s',a'})\end{math}

\item Repeat step 2

\end{enumerate}



The second practical problem arises from the fact that we rarely know the transition

probability for the actions and rewards matrix. Remember what interface provides

Gym to the agent’s writer: we observe the state, decide on an action and only

then do we get the next observation and reward for the transition. We don’t know

(without peeking into Gym’s environment code) what the probability is to get into

state \begin{math}s_1\end{math} from state \begin{math}s_0\end{math} by issuing action

\begin{math}a_0\end{math}. What we do have is just the history from

the agent’s interaction with the environment. However, in Bellman’s update, we

need both a reward for every transition and the probability of this transition. So,

the obvious answer to this issue is to use our agent’s experience as an estimation

for both unknowns. Rewards could be used as they are. We just need to remember

what reward we’ve got on transition from \begin{math}s_0\end{math} to \begin{math}s_1\end{math},

using action a, but to estimate

probabilities we need to maintain counters for every tuple \begin{math}(s_0, s_1, a)\end{math} and normalize

them.

\end{document}

